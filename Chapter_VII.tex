\section{}
\begin{problem}{VII.1.1}
	Prove that if $k \subseteq K$ is a field extension, then $\operatorname{char} k = \operatorname{char} K$. Prove that
	the category $\mathsf{Fld}$ has no initial object.
\end{problem}
\begin{proof}
	Since there is are unique ring-homomorphisms $i : \Bbb Z \to k$ and $j : \Bbb Z \to K$, und the inclusion $\bar i : k \subseteq K$ is a field-homomorphism hence a ring-homomorphism we get $j = \bar i \circ i$. Since $\bar i$ is injective, $\ker \bar i = 0$ and $\ker j = \ker \bar i \circ i = \ker i$. Therefore $k$ and $K$ have the same characteristic.
	If there would exist an initial object in $\mathsf{Fld}$, all fields would have its characteristic by the previous. This is not the case, since $\Bbb Q$ and $\Bbb F_2$ don not have the same  characteristic.
\end{proof}
\begin{problem}{VII.1.4}
	Let $k \subseteq k(\alpha)$ be a simple extension, with $\alpha$ transcendental over $k$. Let $E$ be a subfield of $k(\alpha)$ properly containing $k$. Prove that $k(\alpha)$ is a finite extension of $E$.
\end{problem}

\begin{proof}
	Since $\alpha$ is transcendental over $k$, the field $k(\alpha)$ is isomorphic to the field of rational functions $k(t)$ in one variable via the evaluation isomorphism sending $t$ to $\alpha$. 
	
	Because $E \supsetneq k$, there exists a nonconstant element $f(\alpha) \in E \setminus k$. Write $f(\alpha) = p(\alpha)/q(\alpha)$ in lowest terms, with $p,q \in k[x]$. Consider the polynomial $F(X) = p(X) - f(\alpha) q(X)$ with coefficients in $E$. Substituting $X = \alpha$ gives $F(\alpha) = 0$, so $\alpha$ is algebraic over $E$. 
	
	It follows that $k(\alpha)$ is a finite algebraic extension of $E$.
\end{proof}
\begin{problem}{VII.1.5}
	\medspace
	\begin{itemize}
		\item Prove that there is exactly one subfield of $\mathbb{R}$ isomorphic to $\mathbb{Q}[t]/(t^2 - 2)$.
		
		\item Prove that there are exactly three subfields of $\mathbb{C}$ isomorphic to $\mathbb{Q}[t]/(t^3 - 2)$.
	\end{itemize}
	
	From a ``topological'' point of view, one of these copies of 
	$\mathbb{Q}[t]/(t^3 - 2)$ looks very different from the other two:
	it is not dense in $\mathbb{C}$, but the others are.
\end{problem}
\begin{proof}
	(1) Any subfield of $\Bbb R$ isomorphic to $\Bbb Q[t]/(t^2-2)$ must be generated by a real root of $t^2-2$, hence by $\pm\sqrt{2}$.  But 
	\[
	\Bbb Q(\sqrt{2})=\Bbb Q(-\sqrt{2}),
	\]
	so both choices generate the same subfield of $\Bbb R$.  Since $t^2-2$ has no other real roots, there is exactly one such subfield of $\Bbb R$.
	
	\medskip
	
	(2) The polynomial $t^3-2$ has three distinct complex roots.  Writing
	\[
	2 = 2 e^{2\pi i \cdot 0},
	\]
	de Moivre's formula gives the cube roots
	\[
	\sqrt[3]{2}\, e^{2\pi i k/3},
	\qquad k=0,1,2.
	\]
	These are
	\[
	\sqrt[3]{2}, \qquad 
	\sqrt[3]{2}\,\zeta_3, \qquad 
	\sqrt[3]{2}\,\zeta_3^2,
	\]
	where $\zeta_3=e^{2\pi i/3}$ is a primitive third root of unity.  
	Each root $\alpha$ determines a subfield $\Bbb Q(\alpha)\subseteq \Bbb C$
	isomorphic to $\Bbb Q[t]/(t^3-2)$.  Since the polynomial is separable,
	these roots are distinct, and if 
	\[
	\Bbb Q(\alpha_i)=\Bbb Q(\alpha_j),
	\]
	then $\alpha_i/\alpha_j\in\Bbb Q$, which would force 
	$\zeta_3^k\in\Bbb Q$, impossible for $k=1,2$.  
	Hence the three roots generate three distinct subfields of $\Bbb C$.
	Let $\alpha_0=\sqrt[3]{2}\in\Bbb R$ and 
	\[
	\alpha_k=\sqrt[3]{2}\,\zeta_3^k \qquad (k=1,2),
	\]
	where $\zeta_3=e^{2\pi i/3}$.  As above, the three subfields of $\Bbb C$
	isomorphic to $\Bbb Q[t]/(t^3-2)$ are $\Bbb Q(\alpha_0)$,
	$\Bbb Q(\alpha_1)$, and $\Bbb Q(\alpha_2)$.
	
	The field $\Bbb Q(\alpha_0)=\Bbb Q(\sqrt[3]{2})$ is contained in $\Bbb R$.
	Since $\Bbb R$ is closed in $\Bbb C$ (with the usual topology),
	the closure of $\Bbb Q(\alpha_0)$ in $\Bbb C$ is contained in $\Bbb R$.
	Hence it cannot be dense in $\Bbb C$.
	
	Now let $k=1$ or $2$.  Then $\alpha_k\notin\Bbb R$, so 
	$\Bbb Q(\alpha_k)$ contains a non-real element.  Write
	\[
	\alpha_k = a+ib \quad (b\neq 0).
	\]
	Since $\alpha_k^3=2\in\Bbb R$, one checks that also $\overline{\alpha_k}
	=\alpha_k^2/\sqrt[3]{2}$ lies in $\Bbb Q(\alpha_k)$, hence
	\[
	\Re(\alpha_k)=\frac{\alpha_k+\overline{\alpha_k}}{2},
	\qquad
	\Im(\alpha_k)=\frac{\alpha_k-\overline{\alpha_k}}{2i}
	\]
	belong to $\Bbb Q(\alpha_k)$.  Thus $\Bbb Q(\alpha_k)$ contains a real
	number and an imaginary number that are $\Bbb Q$-linearly independent,
	so it contains elements arbitrarily close to any complex number. Since $\Bbb Q$ is dense
	in $\Bbb R$, it follows that $\Bbb Q(\alpha_k)$ is dense in $\Bbb C$.
\end{proof}


\section{}
\begin{problem}{VII.2.12} 
	Let $K$ be an infinite field. A polynomial function on an affine algebraic set $S \subseteq A_K^n$ is the restriction to $S$ of (the evaluation function of) a polynomial $f(x_1,...,x_n) \in K[x_1,...,x_n]$. Polynomial functions on an algebraic $S$ manifestly form a ring and in fact a $K$-algebra. Prove that this $K$-algebra is isomorphic to the coordinate ring of $S$. 
\end{problem}
\begin{proof}
	Let $\Phi : K[x_1,\dots,x_n] \to \mathcal{P}(S)$ be given by $\Phi(f)=f|_S$. Then $\Phi$ is a surjective $K$-algebra homomorphism by definition of polynomial functions on $S$, and $\ker\Phi = I(S)$, the ideal of polynomials vanishing on $S$. By the First Isomorphism Theorem we obtain
	\[
	K[x_1,\dots,x_n]/I(S) \cong \mathcal{P}(S),
	\]
	which is precisely the coordinate ring $K[S]$. 
\end{proof}
\begin{problem}{VII.3.2}
	Prove that if $a, b$ are constructible numbers, then so is $a−b$.
\end{problem}
\begin{proof}
	If $a$ and $b$ are constructible,
	we may lay off segments of lengths $a$ and $b$ on a common line $L$.
	Construct the circle with center at the point representing $a$ and
	radius $b$.  Its intersection with $L$ on the segment between $0$
	and $a$ has distance $a-b$ from $0$.  Hence $a-b$ is constructible.

\begin{center}
	\begin{tikzpicture}[scale=0.5]

		\draw[->] (-8,0) -- (8,0) node[right] {$L$};
		
		\coordinate (O) at (0,0);
		\coordinate (A) at (4,0);
		\coordinate (B) at (5.5,0);
		\coordinate (M) at (-1.5,0);   
		
		\draw (A) circle (5.5);
		\node[right] at (A) {};
		
		\fill (M) circle (5pt);
		\node[below] at (M) {$a-b$};
		
		\fill (A) circle (5pt);
		\node[below] at (A) {$a$};
		\fill (B) circle (5pt);
		\node[below] at (B) {$b$};
	
		\fill (O) circle (5pt);
		\node[below] at (O) {$0$};
		
	\end{tikzpicture}
\end{center}
\end{proof}
\section{}
\section{}
\begin{problem}{VII.4.1}
	Let $k$ be a field, $f(x)\in k[x]$, $F$ its splitting field over $k$, and 
	$k\subseteq K$ an extension over which $f$ splits. 
	Show that there exists a $k$-homomorphism $F\to K$.
\end{problem}

\begin{proof}
	Let $\alpha_1,\dots,\alpha_r\in K$ be the roots of $f$ in $K$ and set 
	$E := k(\alpha_1,\dots,\alpha_r)\subseteq K$.  
	Then $E$ is a splitting field of $f$ over $k$.  
	By uniqueness of splitting fields up to $k$-isomorphism, there exists a $k$-isomorphism 
	\[
	\varphi : F \xrightarrow{\sim} E .
	\]
	Let $\iota:E\hookrightarrow K$ be the inclusion.  
	Then $\iota \circ \varphi : F\to K$ is a $k$-homomorphism extending $\mathrm{id}_k$.
	
	\[
	\begin{tikzcd}
		k \arrow[r, hook]  \arrow[rd, hook]  & E = k(\alpha_1,\dots,\alpha_r)   \arrow[r, hook, "\iota"] & K \\
		& F \arrow[ru, "\iota \circ \varphi"'] \arrow[u, "\sim", "\varphi"']
	\end{tikzcd}
	\]
\end{proof}
\begin{problem}{VII.4.3}
	Find the order of the automorphism group of the splitting field of $x^4 + 2$
	over $\Bbb Q$ (cf. Example 4.6).
\end{problem}

\begin{proof}
	In the splitting field, the polynomial falls into linear factors, so $|\mathrm{Aut}_{\Bbb Q}(K)|$ is equal to $[K : \Bbb Q]$.	The roots of $x^4+2$ are $\sqrt[4]{2}\,\zeta$, where $\zeta$ runs through the
	fourth roots of $-1$, hence they lie in $\mathbb{Q}(i,\sqrt[4]{2})$.  
	Thus $K=\mathbb{Q}(i,\sqrt[4]{2})$ is a splitting field of $x^4+2$ over $\mathbb{Q}$.
	
	The minimal polynomial of $i$ over $\mathbb{Q}$ is $x^2+1$, so
	\[
	[\mathbb{Q}(i):\mathbb{Q}] = 2.
	\]
	The element $\sqrt[4]{2}$ has the minimal polynomial $x^4-2$ over $\mathbb{Q}$,
	which is irreducible by Eisenstein at $2$, hence
	\[
	[\mathbb{Q}(\sqrt[4]{2}):\mathbb{Q}] = 4.
	\]
	Since $x^4-2$ remains irreducible over $\mathbb{Q}(i)$ (it has no root in $\mathbb{Q}(i)$),
	we obtain
	\[
	[K:\mathbb{Q}(i)] = 4.
	\]
	By the Tower Property (Proposition~1.10),
	\[
	[K:\mathbb{Q}] = [K:\mathbb{Q}(i)]\,[\mathbb{Q}(i):\mathbb{Q}]
	= 4\cdot 2 = 8.
	\]
\end{proof}
\section{}
\section{}
\begin{problem}{VII.6.1}
	Prove Lemma 6.3.
\end{problem}

\begin{proof}
	Let $k \subseteq F$ be a field extension.
	
	\medskip
	
	\textbf{(1) Inclusion-reversing.}  
	If $E_1 \subseteq E_2$ are intermediate fields, then every automorphism fixing $E_2$
	also fixes $E_1$, hence
	\[
	\operatorname{Aut}_{E_2}(F) \subseteq \operatorname{Aut}_{E_1}(F).
	\]
	If $G_1 \subseteq G_2 \subseteq \operatorname{Aut}_k(F)$, then any element fixed by $G_2$
	is fixed by $G_1$, hence
	\[
	F^{G_2} \subseteq F^{G_1}.
	\]
	
	\medskip
	
	\textbf{(2) $E \subseteq F^{\operatorname{Aut}_E(F)}$.}  
	If $\alpha \in E$ and $g \in \operatorname{Aut}_E(F)$, then $g(\alpha)=\alpha$,
	so $\alpha \in F^{\operatorname{Aut}_E(F)}$.
	
	\medskip
	
	\textbf{(3) $G \subseteq \operatorname{Aut}_{F^G}(F)$.}  
	If $g \in G$ and $\alpha \in F^G$, then $h(\alpha)=\alpha$ for all $h \in G$,
	in particular for $h=g$, hence $g$ fixes $F^G$ pointwise.
	
	\medskip
	
	\textbf{(4) $\operatorname{Aut}_{E_1E_2}(F) = \operatorname{Aut}_{E_1}(F)\cap \operatorname{Aut}_{E_2}(F)$.}  
	An automorphism fixes the compositum $E_1E_2$ iff it fixes both $E_1$ and $E_2$,
	which is equivalent to lying in both subgroups.
	
	\medskip
	
	\textbf{(5) $F^{\langle G_1,G_2\rangle} = F^{G_1}\cap F^{G_2}$.}  
	An element is fixed by the subgroup generated by $G_1$ and $G_2$
	iff it is fixed by every element of $G_1$ and of $G_2$,
	which is equivalent to belonging to both fixed fields.
	
\end{proof}

\begin{problem}{VII.6.13}
	Let $k \subseteq F$ and $k \subseteq K$ be Galois extensions, and assume 
	$F$ and $K$ are subfields of a common overfield.  
	Prove that $k \subseteq FK$ and $k \subseteq F \cap K$ are Galois extensions.
\end{problem}

\begin{proof}
	Since $F/k$ and $K/k$ are Galois, there exist polynomials 
	$f,g \in k[x]$ such that $F$ is the splitting field of $f$ over $k$
	and $K$ is the splitting field of $g$ over $k$.  
	
	\medskip
	
	\textbf{(1) The compositum $FK$.}  
	The product $h:=fg \in k[x]$ has as roots exactly the roots of $f$ and of $g$.  
	Hence the splitting field of $h$ over $k$ is the field generated by all roots
	of $f$ and $g$, which is precisely $FK$.  
	Therefore $FK/k$ is a splitting field of a polynomial in $k[x]$, hence Galois.
	
	\medskip
	
	\textbf{(2) The intersection $F \cap K$.}  
	Let $L=F\cap K$.  
	Because $F/k$ and $K/k$ are normal and separable, any irreducible polynomial
	over $k$ having one root in $L$ splits completely in both $F$ and $K$, hence
	in their intersection $L$.  
	Thus $L/k$ is normal; separability is inherited from $F/k$.  
	Therefore $k \subseteq F\cap K$ is Galois.
\end{proof}

\section{}
\begin{problem}{VII.7.3}
	Let $G$ be a finite group. Show that there exists a Galois extension 
	$k \subseteq F$ such that $\operatorname{Aut}_k(F) \cong G$.
\end{problem}

\begin{proof}
	Let $|G|=n$. By Cayley’s theorem, $G$ embeds into $S_n$.  
	Let $K$ be any field and $t_1,\dots,t_n$ algebraically independent over $K$, and let $s_1,\dots,s_n$ be the elementary symmetric polynomials in the $t_i$.  
	By Lemma 7.5, the extension $K(s_1,\dots,s_n)\subseteq K(t_1,\dots,t_n)$ is Galois with Galois group $S_n$, acting by permuting the $t_i$.  
	Identify $G$ with a subgroup of $S_n$ and let $F:=K(t_1,\dots,t_n)^G$ be its fixed field, and set $k:=K(s_1,\dots,s_n)$.  
	Since $K(t_1,\dots,t_n)/k$ is Galois with group $S_n$, the Galois correspondence yields
	\[
	\operatorname{Aut}_k(F)\cong G.
	\]
	Thus $k\subseteq F$ is a Galois extension with Galois group isomorphic to $G$.
\end{proof}

\begin{problem}{VII.7.5}
	Let $k \subseteq F$ and $k \subseteq K$ be radical extensions, and suppose 
	$F,K$ are subfields of a common overfield.  
	Show that the composite $k \subseteq FK$ is radical.
\end{problem}

\begin{proof}
	Since $k\subseteq F$ and $k\subseteq K$ are radical, there exist chains
	\[
	k=F_0 \subseteq F_1 \subseteq \dots \subseteq F_r=F,
	\qquad
	k=K_0 \subseteq K_1 \subseteq \dots \subseteq K_s=K,
	\]
	with $F_i=F_{i-1}(\alpha_i)$ and $\alpha_i^{m_i}\in F_{i-1}$, and similarly
	$K_j=K_{j-1}(\beta_j)$ with $\beta_j^{n_j}\in K_{j-1}$.  
	Consider the interlaced chain
	\[
	k \subseteq F_1 \subseteq \dots \subseteq F_r 
	\subseteq F_rK_1 \subseteq \dots \subseteq F_rK_s=FK.
	\]
	Each step adjoining $\alpha_i$ is radical by assumption, and each step from
	$F_rK_{j-1}$ to $F_rK_j=F_rK_{j-1}(\beta_j)$ is radical because
	$\beta_j^{n_j}\in K_{j-1}\subseteq F_rK_{j-1}$.  
	Hence every extension in the chain is obtained by adjoining an element whose
	power lies in the previous field.  
	Therefore $k\subseteq FK$ is a radical extension.
\end{proof}